\section{Compétences}

\subsection{Système}
\cvline{}{Maîtrise des environnements Linux (Debian et Red Hat), Unix}

\subsection{Langages}
\cvline{}{Expert en Shell, bonnes connaissances en HTML, Python, \LaTeX, PHP \& SQL.}

\subsection{Outils}
\cventrybis{
    \begin{itemize}
        \item{Administration de :}
            \begin{itemize}
                \item{Ansible, AWX, Tower}
                \item{IBM Workload Scheduler (anciennement TWS),}
                \item{OVM, VirtualBox.}
            \end{itemize}
    \end{itemize}
}
\cventrybis{
    \begin{itemize}
        \item{Utilisation de :}
            \begin{itemize}
                \item{Git (GitHub, GitLab, Framagit), }
                \item{Puppet},
                \item{Proxmox, KVM,}
                \item{Jira, Confluence, Bitbucket,}
                \item{CFT.}
            \end{itemize}
    \end{itemize}
}

\subsection{SGBD}
\cvline{}{Connaissances de l'administration de PostgreSQL, MySQL, Oracle}

\subsection{Formations suivies}
\cvlistitem{Ansible : \textit{Automation with Ansible (DO407)} en 2018 par Red Hat. (4 jours)}
\cvlistitem{AGILE : \textit{Introduction et pratique de la méthode AGILE \& SCRUM} en 2013 par Intelli'N. (2 jours)}
\cvlistitem{Oracle : \textit{Introduction technique à Oracle} \& \textit{Administration Oracle (niveau 1)} en 2006 par or@tech. (1+5 jours)}
\cvlistitem{Linux : \textit{Administration Linux Red Hat} en 2003 par Pythagore F.D (10 jours)}

\subsection{Formation dispensée}
\cvlistitem{Linux : \textit{Préparation à la certification Linux LPIC101 \& 102} en 2012 pour Intelli'N (2 + 4,5 jours)}

\subsection{Langue}
\cvline{}{Anglais technique : lu, écrit, traduit et parlé et chanté en solo.}

\newpage

\section{Certification}
\cventry{2011}{LPIC 1}{LPIC 101 \& 102}{}{}{vérification \mbox{\url{https://cs.lpi.org/caf/Xamman/certification}}
\newline{}code LPI = LPI000228139
\newline{}code de vérification = kanfzh9uuj
}
\section{Expériences}

\cventry{}{Activement bénévole}{}{}{}{}
%\cventrybis{
    \begin{itemize}
    	\item{Depuis 2024, fondateur et président du club de tennis de table d’Orry-la-Ville \httpslink[CO Ping]{co-ping.fr} :}
    	\begin{itemize}
    		\item{Gestion de :}
    		\begin{itemize}
    			\item{ses 48 membres}
    			\item{dont 18 jeunes de 8 à 15 ans}
    			\item{et 15 compétiteurs}
    			\item{pour 3 équipes, de 4 membres, dont deux championnes de leur poule en D4 et une nouvelle équipe en janvier 2025.}
    		\end{itemize}
    		\item{Relation avec les mairies, la fédération, la ligue, le comité et les autres clubs}
    		\item{Création et animation d’actions de promotion du club et du tennis de table.}
    		\item{Mise en place de l’informatique nécessaire à la gestion du club (Paheko, Joomla!, NextCloud).}
		\end{itemize}
		\item{Bénévole, sur une semaine, au WTT (World Table Tennis) Champions de Montpellier en octobre 2024 : guide des spectateurs et chaperon des sportifs et membres dirigeants de la WTT.}
		\item{Pour l'association \httpslink[Orry en Transition]{orryentransition.fr/}}
            \begin{itemize}
                \item{Préparation et animation d'une rencontre sur les logiciels libres et la sobriété numérique (20 janvier 2022)}
                \item{Recyclage d'ordinateurs avec l'utilisation de Linux Debian,}
                \item{Actions de promotion de l'association : troc aux plantes, réparation de vélos, création d'un hôtel à insectes, nettoyage de la ville et des forêts attenantes\dots}
            \end{itemize}
        \item{\httpslink[Oisux]{www.oisux.org}, association de promotion des Logiciels Libres sur le département de l'Oise.}
            \begin{itemize}
                \item{Assistance informatique depuis 2007,}
                \item{Promotion de la distribution Primtux au salon Open Source Expérience et au Carnaval des Possibles.}
            \end{itemize}
        \item{\httpslink[Orry'zon]{www.orryzon.fr} association culturelle.}
        \item{\httpslink[ArtemOise]{www.artemoise.fr} association culturelle de promotion de musiques lyriques.}
            \begin{itemize}
                \item{Pour la bonne réalisation d'une balade musicale,}
                \item{Pour l'organisation de la 5e édition des Orryginales, concentration de voitures d'exception.}
                \item{Machiniste pour des spectacles musicaux :}
                    \begin{itemize}
                        \item{\textit{Les Trois Ténors aux J.O}, à Chantilly, le 22 juin 2024,}
                        \item{\textit{Véronique}, à Coye-la-Forêt, le 29 septembre 2024,}
                        \item{\textit{Le Secret de Suzanne}, à Orry-la-Ville, le 15 décembre 2024,}
                        \item{\textit{Noël se fête}, à Orry-la-Ville, le 22 décembre 2024.}
                    \end{itemize}
            \end{itemize}
        \item{Pour le festival de cinéma \httpslink[Orry Film Festival]{www.orryfilmfestival.fr} d’Orry-la-Ville,}
            \begin{itemize}
                \item{Sa préparation,}
                \item{Visionnage et sélection de films,}
                \item{Présentation des films pour l'édition 2022,}
                \item{Animation des débats post projection.}
            \end{itemize}
        \item{Pour le \textit{repair café} \httpslink[Rep'aire au vert]{www.repaircafe.org/fr/cafe/repair-cafe-lamorlaye-et-ses-environs/} de Lamorlaye}
            \begin{itemize}
                \item{Petites réparations informatiques}
                \item{Sensibilisation autour des logiciels libres}
            \end{itemize}
        \item{Cueillette de plantes médicinales pour \httpslink[Les Plantes de Mathilde]{lesplantesdemathilde.fr/}}.
        \item{Rédaction (en Markdown et \LaTeX) et animation d'une conférence sur Led Zeppelin (\httpslink[à lire ici]{iels-ledub.netlib.re/site/musique/}).}
        \item{Participation à la rédaction de la formation Linux \httpslink[Formatux]{www.formatux.fr}}
    \end{itemize}
%}
\cvdoubleitem{Système :}{Debian, CentOS}{}{}
\cvdoubleitem{Langages :}{Shell, Markdown, \LaTeX }{Outils :}{Ansible, Git, Pandoc, Proxmox, Yunohost, sécateur\dots}

%%% Expériences PRO

\cventry{2022--2024~\\25 mois}{Administrateur système Linux N3}{Société Générale}{Val-de-Fontenay}{Orness}{\textbf{Administrateur système niveau 3}}
\cventrybis{
    \begin{itemize}
        \item{Maintien en condition opérationnelle des serveurs physiques utilisés pour le trading à haute fréquence.}
        \item{Amélioration et simplification des tests de validation d’état de santé des serveurs par une réécriture, en de multiple scripts Shell, d'un gros script Python.}
        \item{Rédaction de documentations relatives à mon activité.}
        \item{Relation avec les fournisseurs et les partenaires qui gèrent les différents hébergements.}
        \item{Configuration de BIOS/iLO et système d’exploitation Red Hat 7 \& 8.}
        \item{Configuration de modules Puppet pour la création de serveurs en Red Hat 8.}
        \item{Masterisation de serveurs en Red Hat 8 ainsi que leur configuration définitive}
        \item{Tout cela dans un environnement anglophone en relation avec des collègues Indiens et des prestataires étrangers}
    \end{itemize}
}

\cvdoubleitem{Système :}{Red Hat}{}{}
\cvdoubleitem{Langages :}{Shell, Python}{Outils :}{Puppet, Ansible}   %% SoGé
\cventry{2022--2022~\\6 mois}{Administrateur système Linux N2/N3}{SNCF}{Lyon/La Défense}{Modis}{\textbf{Administrateur système niveau 2 \& 3 :}}
\cventrybis{
	\begin{itemize}
		\item{Analyse de la performance de systèmes Linux et recherche de l'origine de leur charge machine,}
		\item{Utilisation de DataDog, Centreon et commandes Unix (\texttt{top}, \texttt{sar}, \texttt{pidstat}\dots{}),}
		\item{Gestion des ressources de VM depuis Vmware ESX : ajout d'espace disque, RAM, mémoire,}
		\item{Action sur LVM : création, augmentation et diminution de partitions,}
	\end{itemize}
}
\cventrybis{
	\begin{itemize}
		\item{Administration de serveurs dans le cloud Azure (configuration de \textit{Load Balancer}, ajout de ressources CPU et disque,}
		\item{Rédaction de notes pour la gestion de LVM et des différentes commandes d'analyse système (\texttt{top}, \texttt{iostat}, \texttt{pidstat}\dots{})}
	\end{itemize}
}
\cventrybis{
	\textbf{Administrateur DBA Oracle niveau 1 :}
	\begin{itemize}
		\item{Analyse des dysfonctionnements de connexion Oracle,}
		\item{Gestion des ressources disques pour les \textit{datafiles},}
		\item{Création de dump.}
	\end{itemize}
}
\cventrybis{
	\textbf{Administrateur applicatif :}
	\begin{itemize}
		\item{Application de consignes pour la reprise de traitements depuis Dollar Universe}
		\item{Présence à des réunions et actions pour la résolution d'incidents, majeurs ou récurrents}
		\item{Tentative d'amélioration et de correction de consignes, de scripts et de la gestion des documentations}
	\end{itemize}
}

\cvdoubleitem{Système :}{Red Hat, Ubuntu}{Cloud :}{Azure}
\cvdoubleitem{Langages :}{Shell}{Outils :}{ServiceNow, DataDog, Centreon} %% SNCF
\cventry{2022--2022~\\1,5 mois}{Ingénieur DevOps Linux}{EU-Lisa}{Strasbourg}{Next-Ventures}{Dans un environnement exclusivement anglophone}
\cventrybis{
	\begin{itemize}
		\item{Rédaction de playbook et de rôles pour la gestion de Pod Kubernetes}
		\item{Assistance à la rédaction de scripts d'administration applicative}
	\end{itemize}
}
\cvdoubleitem{Système :}{Red Hat}{}{}
\cvdoubleitem{Langages :}{Shell}{Outils :}{Ansible, Kubernetes, Git} %% EU-Lisa
\newpage
\cventry{2019--2021~\\19 mois}{Ingénieur Infra Linux}{Bforbank}{La Défense}{SSIELL}{}
\cventrybis{
    \begin{itemize}
        \item{Gestion de migration de serveurs Red Hat 6.6 en 7.7}
            \begin{itemize}
                \item{Planification}
                \item{Analyse de l'existant}
                \item{Homogénéisation des scripts}
                \item{Amélioration de l'existant}
            \end{itemize}
        \item{Maintien en condition opérationnelle d'une centaine de serveurs}
        \item{Gestion des demandes auprès de l'hébergeur}
        \item{Création de \textit{playbooks} Ansible pour la création d'un socle Linux commun à tous les serveurs}
        \item{Gestion de dépôts Git}
        \item{Validation des évolutions des scripts}
        \item{Support des changements IWS}
        \item{Rédaction, mise à jour de documentations sur IWS, CFT, Git et l'administration Linux}
    \end{itemize}
}

\cvdoubleitem{Système :}{Red Hat}{}{}
\cvdoubleitem{Langages :}{Shell}{Outils :}{Ansible, IWS, Git, Tower.}
   %% Bforbank Ingé infra
\cventry{2018--2019~\\18 mois}{Gestionnaire de postes Linux}{STET}{La Défense}{SSIELL via BK OSI}{Administrateur Linux}
\cventrybis{
    \begin{itemize}
        \item{Administration des postes Linux Ubuntu avec Ansible}
            \begin{itemize}
                \item{Rédaction et amélioration des \textit{PlayBooks} Ansible}
                \item{Installation, configuration et maintien à jour des postes Linux Ubuntu}
                \item{Amélioration de la sécurité des postes (\textit{Hardening})}
            \end{itemize}
        \item{Inventaire des configurations, des postes et pièces détachées}
        \item{Réalisation, avec Proxmox, et mise en place d'un serveur PXE et d'un serveur de paquets \texttt{apt-cacher-ng}}
    \end{itemize}
}
\cventrybis{
    \begin{itemize}
        \item{Rédaction de documentations sur les \textit{PlayBooks} et l'administration et l'utilisation des serveurs PXE et \texttt{apt-cacher-ng}}
        \item{Administrateur d'AWX, Jira et Confluence de Atlassian}
    \end{itemize}
}

\cvdoubleitem{Système :}{Ubuntu, Debian}{}{}
\cvdoubleitem{Langages :}{Shell, Python}{Outils :}{Ansible, AWX, Promox}
   %% STET
\cventry{2018~\\5 mois}{Administrateur déploiements applicatifs}{Generali}{Saint-Denis}{SSIELL via Apside}{Administration de la solution Serena de Micro Focus}
\cventrybis{
    \begin{itemize}
        \item{Administration de la solution}
        \item{Ajustement de la configuration des outils de déploiements}
        \item{Gestion de configuration Ansible}
        \item{Modification de scripts Shell pour la gestion d'un dépôt de fichiers et pour la parallélisation des commandes JBoss}
        \item{Correction de scripts Groovy}
    \end{itemize}
}

\cvdoubleitem{Système :}{Red Hat}{}{}
\cvdoubleitem{Langages :}{Shell, Groovy}{Outils :}{Ansible, SBM, SDA}
   %% Générali
\newpage
\cventry{2014--2018~\\42 mois}{Administrateur IWS, expert Shell}{BforBank}{La Défense}{SSIELL}{}
\cventrybis{
    \begin{itemize}
        \item{Administration de IWS :}
        \begin{itemize}
            \item{Montée de version de l'ordonnanceur IWS, moteurs et agents, de la version 8.5 à la 9.2,}
            \item{Migration des moteurs de Windows\texttrademark vers Linux Red Hat,}
            \item{Définition et mise en place de nouvelles normes des objets IWS et de leur gestion dans Git,}
            \item{Rédaction de la procédure de livraison des objets IWS,}
            \item{Création et maintenance des scripts d'administration et de surveillance (Centreon),}
            \item{Création de chaînes complexes (conditions multiples, gestion d'événements, etc\dots),}
            \item{Analyse de la performance de la production informatique (dérive des temps de traitements, espace disque, taille des fichiers échangés\dots),}
            \item{Gestion des calendriers.}
        \end{itemize}
    \end{itemize}
}
\cventrybis{
    \begin{itemize}
        \item{Ingénieur de production :}
        \begin{itemize}
            \item{Rédaction de documentations,}
            \item{Recherche et traitement sur des fichiers (\texttt{find}, \texttt{sed}, \texttt{awk} \dots{},)}
            \item{Suivi de la production : reprise de traitements, corrections de chaînes et/ou de scripts,}
            \item{Relations avec les autres équipes autant internes qu'externes (IBM et autres partenaires de la banque),}
            \item{Sur un parc d'une centaine de serveurs, premier intervenant pour l'analyse des dysfonctionnements système.}
        \end{itemize}
    \end{itemize}
}
\cventrybis{
    \begin{itemize}
        \item{Expert Shell :}
        \begin{itemize}
            \item{Gestion des versions des scripts par Git,}
            \item{Modification et création de scripts,}
            \item{Rédaction de bonnes pratiques à l'utilisation de serveurs Linux,}
            \item{Création de scripts de contrôle de fichiers par Sentinel ou GateWay, d'interrogation Oracle et MySQL, de sauvegarde MySQL et PostgreSQL, et d'inter-action avec le logiciel Core BanKing T24 (Temenos),}
            \item{Animation de micro-formations dispensées auprès de mes collègues sur Unix \& Linux, Shell, Logiciels libres,}
        \end{itemize}
    \end{itemize}
}

\cvdoubleitem{Système :}{Red Hat}{SGBD :}{Oracle, DB2, MySQL, PostgreSQL.}
\cvdoubleitem{Langages :}{Shell, SQL}{Outils :}{IWS, CFT, Sentinel, GateWay, Centreon, Git.}   %% Bforbank Ingé prod
\cventry{2013--2014~\\12 mois}{Administrateur Système Linux, DBA junior}{Cour des Comptes}{Paris}{}{}
\cventrybis{
    \begin{itemize}
        \item{Mise en place d'un serveur \textit{Debian} pour la :}
            \begin{itemize}
                \item{Mise en place d'une base documentaire avec \textit{Dokuwiki},}
                \item{d'une centralisation des mots de passe avec \textit{KeePassX},}
                \item{Mise à jour de \textit{Nagios}.}
            \end{itemize}
    \end{itemize}
}
\cventrybis{
    \begin{itemize}
        \item{Administration de bases de données \textit{Oracle} \& \textit{MySQL} :}
            \begin{itemize}
                \item{Rédaction de documentations relatives à l'administration des applications, du système Linux et bases de données.}
                \item{Rédaction de consignes et de procédures pour la gestion des applications.}
            \end{itemize}
    \end{itemize}
}
\cventrybis{
    \begin{itemize}
        \item{Rédaction de scripts :}
            \begin{itemize}
                \item{d'analyse des \textit{LUN NetApp} et \textit{IBM},}
                \item{de cartographie réseau avec \texttt{nmap}.}
            \end{itemize}
    \end{itemize}}
\cventrybis{
    \begin{itemize}
        \item{Administration du serveur Oracle VM,}
            \begin{itemize}
                \item{Administration de serveurs Oracle Linux sous Oracle VM,}
                \item{Création de squelettes Oracle Linux sur \textit{VMware}}
                \item{Installation et configuration d'un serveur \textit{Oracle VM Server, OVS}.}
                \item{Création d'un prototype \textit{Nagios} sur \textit{CentOS} et \textit{Debian}}
            \end{itemize}
    \end{itemize}
}

\cvdoubleitem{Système :}{Oracle Linux, CentOS, Debian}{SGBD :}{Oracle, MySQL}
\cvdoubleitem{Langages :}{Shell, SQL}{Outils :}{OVM, VMware, TINA}
   %% Cour des Comptes
\cventry{2012~\\6 mois}{Expert TWS}{Véolia}{Saint-Maurice}{SSIELL via Logware}{Migration TWS 8.3 vers un TWS 8.6 (210 workstations)}
\cventrybis{
	\textbf{Expert TWS} :
		\begin{itemize}
			\item{Création de scripts Unix et SQL (Oracle) pour réaliser un inventaire des objets TWS et supprimer les obsolètes,}
			\item{Analyse des ressources systèmes pour l'installation de TWS 8.6,}
			\item{Rédaction de feuille de route de la migration,}
			\item{Animation de réunions de travail autour de la migration,}
			\item{Proposition d'organisation d'une production avec TWS,}
			\item{Validation de la migration par de nombreux tests,}
			\item{Rédaction de scripts pour la migration des données TWS,}
			\item{Rédaction de la procédure de migration,}
		\end{itemize}
	\textbf{Administrateur applicatif} :
		\begin{itemize}
			\item{Rédaction de scripts et de consignes}
		\end{itemize}
}

\cvdoubleitem{Système :}{Linux Red Hat \& Suse}{SGBD :}{Oracle}
\cvdoubleitem{Langages :}{Shell}{Outils :}{TWS}
   %% Véolia
\cventry{2012~\\6,5 jours}{Formateur Linux}{Intelli'n}{Soissons}{SSIELL}{Formateur, en binôme}
\cventrybis{
    \begin{itemize}
        \item{9 élèves candidats à la certification LPIC101 \& LPIC102}
        \item{6 certifiés aux examens 101 \& 102 et 2 au 102}
        \item{2 jours de formation pour le niveau 101}
        \item{4,5 jours pour le niveau 102}
        \item{1/2 journée pour les deux évaluations}
    \end{itemize}
}
   %% Intelli'n
\cventry{2011--2012~\\8 mois}{Expert TWS}{Air France}{Orly}{SSIELL via Proord}{Migration de deux environnements d'un TWS 8.2 vers un TWS 8.5 (environ 200 et 120 workstations)}
\cventrybis{
    \begin{itemize}
        \item{Formation découverte de Linux après d'un collègue Air France (15 heures)}
        \item{Rédaction d'une formation TWS à l'attention de collègues Air France}
        \item{Création de scripts d'aide à la gestion de la production Unix / Linux sur TWS :}
            \begin{itemize}
                \item{récupération des traitements trop fréquemment en erreur}
                \item{automatisation d'envoi de message d'information d'erreur}
                \item{nettoyage des objets obsolètes (calendrier, traitement orphelin, chaîne vide, etc\dots)}
                \item{génération de pages HTML contenant les informations des chaînes et un schéma des enchaînements des traitements}
            \end{itemize}
    \end{itemize}
}

\cvdoubleitem{Système :}{Red Hat}{SGBD :}{DB2}
\cvdoubleitem{Langages :}{Shell}{Outils :}{TWS}
   %% Air France
\cventry{2011--2011~\\6 mois\\(temps partiel)}{Expert TWS}{BforBank}{La Défense}{SSIELL}{Mise en place d'un ETL
\begin{itemize}
    \item{Installation de l'agent TWS sur les nouveaux serveur ETL Powercenter}
    \item{Configuration des moteurs TWS pour gérer ces nouveaux agents}
    \item{Création de scripts de gestion de l'ETL dans la production TWS}
    \item{Rédaction de documentations et formation des exploitants}
\end{itemize}
}

\cvdoubleitem{Système :}{Linux Red Hat}{}{}
\cvdoubleitem{Langages :}{Shell}{Outils :}{TWS, Powercenter}
   %% Bforbank Admin TWS
\cventry{2010--2011~\\7 mois\\(plein temps puis partiel)}{Ingénieur de production}{La Mutualité Française}{Paris XV}{SSIELL via Profilhom}{Analyse et amélioration de l'utilisation d'un LDAP en synchronisation.%}
%\cventrybis{
\begin{itemize}
    \item{Rédaction de documentations relatives à la gestion des incidents de production et des actions de mises à jour des environnements de recette, pré-production et production}
    \item{Gestion des incidents de la production gérée par \$U}
    \item{Écriture de scripts pour :}
        \begin{itemize}
            \item{récupérer les données des deux LDAP}
            \item{faire des statistiques}
            \item{faire des mises en production}
            \item{sauvegarder le LDAP}
        \end{itemize}
\end{itemize}
}

\cvdoubleitem{Système :}{Linux Red Hat}{}{}
\cvdoubleitem{Langages :}{Shell}{Outils :}{\$U, LDAP}
   %% Mutalité Française
\cventry{2009--2010~\\12 mois}{Ingénieur de production}{BforBank}{La Défense}{SSIELL via ITS Group}{Mise en place de la production informatique de la nouvelle banque en ligne BforBank.}
\cventrybis{
    \begin{itemize}
        \item{Rédaction et application des normes de la production : arborescence, nom des scripts, des traces et des objets TWS,}
        \item{Mise en place et administration d'un Wiki,}
        \item{Création, dans ce wiki, d'un tableau de bord de la production c'est-à-dire les exécutions et la configuration des traitements TWS}
        \item{Création de scripts Unix, Linux et batch Windows\texttrademark pour CFT, T24, DB2, Oracle et l'administration système et applicative Unix - Linux}
        \item{Création de documentations relatives à l'administration système et applicative,}
        \item{Administration système des serveurs de tests (Debian, Red Hat, AIX),}
    \end{itemize}
    Expert TWS
    \begin{itemize}
        \item{Mise en place de la production dans TWS}
        \item{Mise en place d'un serveur de secours de TWS}
    \end{itemize}
}
\cventrybis{
    \begin{itemize}
        \item{Création de scripts nécessaires à la sauvegarde et l'écrasement de la base de données DB2 pour le serveur de secours}
        \item{Configuration de ces traitements dans TWS}
        \item{Rédaction d'une procédure de bascule de moteur TWS}
        \item{Formation}
    \end{itemize}
}
\cvdoubleitem{Système :}{Linux Red Hat, Debian, AIX}{}{}
\cvdoubleitem{Langages :}{Shell}{Outils :}{TWS, CFT}
   %% Bforbank Ingé prod
\cventry{2008--2009~\\6 mois}{Développeur SHELL}{Agence pour l'Informatique Financière de l'État}{Noisy le Grand}{SSIELL via Progressive}{Réalisation d'une application de distribution de fichiers sur l'ensemble des serveurs du projet CHORUS.}
\cventrybis{
    \begin{itemize}
        \item{Analyse de l'existent, rédaction des spécifications, écriture de l'application en shell,}
        \item{Rédaction des consignes d'utilisation, test de la solution}
        \item{Amélioration de scripts de sauvegarde de fichiers de configuration des équipements réseaux.}
        \item{Création de scripts pour la génération de fichiers .XLS de la configuration du SAN}
    \end{itemize}
}

\cvdoubleitem{Système :}{Red Hat}{}{}
\cvdoubleitem{Langages :}{Shell}{}{}   %% AIFE
\cventry{2007--2008~\\18 mois}{Ingénieur de production}{SDDC par NetLevel}{Paris III}{}{%}
    %\cventrybis{
    \begin{itemize}
        \item{Rédaction de scripts pour l'analyse de l'exploitation et pour la mise en place de nouvelles chaînes (\textit{jobstream}).}
        \item{Réalisation de scripts d'aide à l'utilisation et à la configuration de TWS.}
        \item{Analyse de l'architecture Unix des applications CFT, TWS \& des Data WareHouses en vue d'une migration des serveurs de SunOS en IBM AIX.}
        \item{Administration système niveau I \& II des serveurs AIX.}
        \item{Création d'une application de surveillance de la dérive du temps des traitements à l'aide de PHP et de MySQL sur un serveur Debian.}
        \item{Création de requêtes SQL pour la surveillance des traitements applicatifs.}
        \item{Mise en place de nouveaux traitements applicatifs.}
        \item{Rédaction de consignes à l'attention des collègues et des pilotes.}
        \item{Support à la rédaction de scripts Unix.}
        \item{Configuration des transferts CFT.}
        \item{Administration de l'ordonnanceur TWS}
        \item{Administration de l'ETL PowerCenter d'informatica}
    \end{itemize}
    }

    \cvdoubleitem{Système :}{SunOs, Debian, AIX.}{SGBD :}{Oracle, MySQL.}
    \cvdoubleitem{Langages :}{Shell, PHP}{Outils :}{TWS, CFT, PowerCenter d'informatica}
   %% SDDC
\cventry{2006--2007~\\6 mois}{Analyse d'exploitation, intégrateur}{HSBC par NetLevel}{La Défense}{}{Mise en place de l'industrialisation des livraisons Unix \& Windows\texttrademark.%}
%\cventrybis{
\begin{itemize}
    \item{Création de scripts ksh et .bat.}
    \item{Gestion des changements.}
    \item{Administration de l'ordonnanceur \$Universe.}
    \item{Diagnostic et résolution d'incidents et problèmes de production.}
    \item{Analyse de performance de nouvelles versions d'applications.}
\end{itemize}
}

\cvdoubleitem{Système :}{SunOs, Suse}{SGBD :}{Oracle, Sybase, Access}
\cvdoubleitem{Langages :}{Shell, Perl}{Outils :}{\$Universe, DBArtisan.}
   %% HSBC
\cventry{2006~\\9 mois}{Ingénieur applicatif Unix}{Natexis-Alta\"ir par NetLevel}{Lognes}{}{Administration d'applications Unix sous SunOs, Hp \& Linux.}
\cventrybis{
    \begin{itemize}
        \item{Développement de scripts pour les mises en production.}
            \begin{itemize}
                \item{}Paramétrage d'outils TSM, Sysload, Monitor.
                \item{}Gestion des changements : applicatif, système.
            \end{itemize}
        \item{Administration d’applications Unix sous SunOs, Hp \& Linux.}
    \end{itemize}
}

\cvdoubleitem{Système :}{ Unix HP, SunOs, Linux.}{SGBD :}{Oracle, Sybase.}
\cvdoubleitem{Langages :}{Shell, Perl.}{Outils :}{Sysload, VmWare, Mon.}   %% Natexis-Altaïr
\cventry{2005~\\12 mois}{Ingénieur d'exploitation Unix}{Roche par NetLevel}{Neuilly sur Seine}{}{Adaptation de l'exploitation Unix, Windows et SAP de l'ordonnanceur Unicenter vers Tivoli Workload Scheduler.}
\cventrybis{
    \begin{itemize}
        \item{Développement de scripts pour la récupération de données depuis TWS vers Excel\texttrademark.}
        \item{Mise en place d'une procédure de demande de changements pour l'ordonnancement.}
        \item{Administration de TWS.}
        \item{Utilisation de SAP.}
    \end{itemize}}

\cvdoubleitem{Système :}{HP-UX}{SGBD :}{Oracle.}
\cvdoubleitem{Langage :}{Shell, Perl}{Outils :}{TWS.}
   %% Roche laboratoire
\newpage
\cventry{2004~\\6 mois}{Analyste d'exploitation Unix}{Cetelem par NetLevel}{St Ouen}{}{Pour la comptabilité, développement et correction de scripts.}
\cventrybis{
    \begin{itemize}
        \item{Analyse / reprise de traitements incidentés, mise en recette et en production.}
        \item{Utilisation de l'ordonnanceur Unicenter de Computer Associates et des sauvegardes par Tivoli Storage Manager.}
        \item{Restauration de fichiers par TSM.}
    \end{itemize}
}

\cvdoubleitem{Système :}{SunOs}{SGBD :}{Oracle.}
\cvdoubleitem{Langage :}{Shell}{Outils :}{Unicenter, Tivoli Storage Manager.}
   %% Cetelem
    \cventry{2003--2004~\\8 mois}{Analyste d'exploitation Unix \& NT}{EDF par Atos}{Clamart}{}{}\cventrybis{
        \begin{itemize}
            \item{Sur Linux, correction de scripts bash pour la surveillance de traces d'exploitation.}
            \item{Utilisation de l'ordonnanceur Control-M et des sauvegardes par Networker.}
            \item{Réalisation d'actions demandées pour l'installation de nouvelles applications.}
        \end{itemize}
    }

    \cvdoubleitem{Système :}{Red Hat, SunOs}{SGBD :}{Oracle client Windows, serveur Unix.}
    \cvdoubleitem{Langage :}{Shell}{Outils :}{Control-M, Networker.}
   %% EDF
\cventry{2001--2003~\\30 mois}{Analyste d'exploitation Unix, NT \& VMS}{Crédit Lyonnais par Atos}{La Défense}{}{Analyse et résolution des incidents applicatifs (Summit, EAI, Murex\dots{}) \& systèmes (HP-UX \& SunOs).}
\cventrybis{
    \begin{itemize}
        \item{Création de consignes,}
        \item{Développement de scripts pour la gestion de comptes Unix \& Sybase et l'exploitation d'un EAI.}
        \item{Amélioration des scripts d'exploitation de l'application Summit.}
        \item{Surveillance Tivoli, Administration de l'ordonnanceur \$Universe et des sauvegardes par Networker.}
        \item{Migration d'une production de HP-UX vers SunOs pour l'application Summit.}
    \end{itemize}
}

\cvdoubleitem{Système :}{HP-UX, SunOs, VMS, NT}{SGBD :}{Sybase, Oracle.}
\cvdoubleitem{Langage :}{Shell, SQL}{Outils :}{Surveillance Tivoli, \$Universe, Networker.}   %% Crédit Lyonnais
\cventry{2000--2001~\\3 mois}{Technicien d'exploitation Unix, NT \& VMS}{Crédit Lyonnais par Atos}{Paris XI}{}{Application de consignes pour la résolution d'incidents de production.}
\cventrybis{
    \begin{itemize}
        \item{Utilisation de la surveillance Tivoli, utilisation de l'ordonnanceur \$Universe et restauration des sauvegardes par Networker.}
        \item{Application des consignes sur remontée d'incident}
    \end{itemize}
}

\cvdoubleitem{Système :}{HP-UX, SunOs, VMS, NT}{SGBD :}{Sybase}
\cvdoubleitem{}{}{Outils :}{Surveillance Tivoli, \$Universe, Networker.}   %% Crédit Lyonnais - pilote
\cventry{2000~\\3 mois}{Technicien d'exploitation Unix}{France Télécom Câble par Seevia}{Malakoff}{}{Surveillance de processus Oracle, gestion des sauvegardes.}
\cventrybis{
    \begin{itemize}
        \item{Administration de l'ordonnanceur \$Universe.}
        \item{Création de scripts pour la gestion de sauvegardes et mises en production.}
    \end{itemize}
}

\cvdoubleitem{Système :}{SunOs}{SGBD :}{Oracle.}
\cvdoubleitem{Langages :}{Shell, HTML}{Outils :}{\$Universe.} %% France Télécom
\newpage
\cventry{2000~\\6 mois}{Technicien informatique Unix}{Riva-Anker}{Nanterre}{}{Migration des systèmes informatiques des Intermarché.}
    \cventrybis{
        \begin{itemize}
            \item{Formateur des techniciens au BSD}
            \item{Support technique}
            \item{Formation des utilisateurs au nouveau système d'encaissements}
        \end{itemize}
    }

    \cvdoubleitem{Système :}{ Unix Sco, BSD}{}{}
    \cvdoubleitem{Langages :}{Shell}{}{}
 %% Riva
\cventry{1996--2000~\\41 mois}{Tech. de maintenance, Responsable d'agence}{Micro-Maintenance-France}{Paris puis Arras}{}{Maintenance des systèmes \& matériels pour pharmacies de ville.}
\cventrybis{
    \begin{itemize}
        \item{Responsable des trois techniciens de l'agence d'Arras.}
        \item{Gestion des stocks et planification des interventions et installations.}
        \item{Dépannage sur site et par téléphone}
        \item{Formation de clients et du personnel.}
    \end{itemize}
}

\cvdoubleitem{Système :}{Interactive Unix, Sco}{}{}
\cvdoubleitem{Langages :}{Shell}{}{}   %% CIP-SILMM / MMF

\closesection{}

\section{Formation : Électronique}
\cventry{1994}{B.T.S}{}{}{}{}
\cventry{1992}{Baccalauréat}{}{}{}{}
\cventry{1990}{B.E.P}{}{}{}{}
\closesection{}

\section{Centres d'intérêts}
\cvline{Informatique}{autodidacte, curieux et passionné de logiciels libres en général et GNU/Linux en particulier,\newline{}
Utilisateur de Linux depuis 1999 et de Debian depuis 2002}
\cvline{L'entraide}{Adhérent d'une association de promotion des Logiciels Libres (Oisux).}
\cvline{Sport}{Pongiste acharné et dopé à l'amitié de ses partenaires.}
%\closesection{}