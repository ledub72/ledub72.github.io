\cventry{2022--2024~\\25 mois}{Administrateur système Linux N3}{Société Générale}{Val-de-Fontenay}{Orness}{}
\cventrybis{
    \begin{itemize}
      \item{Maintien en condition opérationnelle des serveurs physiques (HPE) utilisés pour le trading à haute fréquence.}
      \item{Serveurs \textit{overclockés} et au noyau optimisé, équipés de carte réseau Solarflare.}
      \item{Actions dans l'interface de gestion des serveurs physique HPE (iLO 2 à iLO 5) : analyse des log, diagnostiques, configuration BIOS, récupération d'informations (adresse MAC, configuration matériel).}
      \item{Installation et configuration de Red Hat 7 \& 8 depuis l'iLO à partir d'une iso associée à chaque serveur et le passage d'une action de Puppet.}
      \item{Configuration de modules Puppet pour la création de serveurs en Red Hat 8.}
      \item{Création des paquet du pilote Solarflare et de ses logiciels dépendants pour Red Hat 8.}
      \item{Utilisation d'Ansible pour la partie \textit{morning check}, donc maintien des playbooks (documentation, simplification, amélioration de la gestion de l'inventaire des serveurs).}
      \item{Amélioration et simplification des tests de validation d’état de santé des serveurs par une réécriture, en de multiple scripts Shell, d'un gros script Python.}
      \item{Rédaction de documentations relatives à mon activité.}
      \item{Relation avec les fournisseurs et les partenaires qui gèrent les différents hébergements.}
      \item{Relation avec les équipes réseau, sécurité et fourniture de serveurs de la Société Générale.}
      \item{Tout cela dans un environnement anglophone avec mes collègues Indiens et les prestataires étrangers.}
    \end{itemize}
}
\cvdoubleitem{Système :}{Red Hat}{}{}
\cvdoubleitem{Langages :}{Shell, Python}{Outils :}{Puppet, Ansible}
